\documentclass[aspectratio=169,mathserif,11pt]{beamer}
%\usefonttheme{professionalfonts}
%\usefonttheme{serif}
\usepackage{hyperref}
%\usepackage[T1]{fontenc}
\UseRawInputEncoding

% Cannot enable in Xelatex
\usepackage{pgfpages}
% \setbeameroption{hide notes} % Only slides
% \setbeameroption{show only notes} % Only notes
% \setbeameroption{show notes on second screen}
\usepackage[style=alphabetic, backend=bibtex, giveninits=true, maxbibnames=99]{biblatex}
\DeclareNameAlias{default}{last-first}
\renewbibmacro{in:}{} % Remove "In:" before journal title
\DeclareFieldFormat{pages}{#1} % Remove "pp." before page numbers
\DeclareFieldFormat[article, inbook, incollection, inproceedings, thesis, misc]{title}{#1} % Remove quotation marks around titles
\DeclareFieldFormat[article]{volume}{\mkbibbold{#1}} % Suppress "Volume" for articles
\DeclareFieldFormat[article]{number}{\mkbibparens{#1}}% Suppress "number" for articles
\renewcommand*{\newunitpunct}{\addspace\addcolon\space}% Add a space between volume and number
%\renewbibmacro*{date}{\printdate}% Change the format of the date
\DeclareFieldFormat[book]{edition}{}% Suppress "edition" for books
\AtEveryBibitem{\ifentrytype{incollection}{\clearname{editor}}{}} % Suppress "editor" for incollection
\addbibresource{reference_9_sem.bib}
\usepackage{filecontents}
% other packages
\usepackage{latexsym,amsmath,xcolor,multicol,booktabs,calligra,multirow}
\usepackage{graphicx}
%\usepackage[capbesideposition=right]{floatrow}
\usepackage{rotating}
\usepackage{tikz}
\usetikzlibrary{arrows, arrows.meta}
\usetikzlibrary{patterns, patterns.meta}
\usetikzlibrary {3d}
\usetikzlibrary{snakes}
\usepackage{etoolbox}
\usepackage{tabularx}
\usepackage{array}
\newcolumntype{M}[1]{>{\centering\arraybackslash}m{#1}}
\usepackage{caption}
%\usepackage{subfigure}
\usepackage{subcaption}
%% Enable only in Xelatex
% \usepackage{pstricks}

\title{{\rm Existence, Uniqueness and Regularity of  Solutions to the 3D Navier--Stokes Equations: Understanding Singularities}}
%\subtitle{Presentation}
\author[{\rm Ajay Kumar}]{
\centering
\includegraphics[width=0.1\linewidth]{mnit.jpg}\\
\vspace{0.1cm}
Ajay Kumar\\ {\footnotesize(2025RMA9073)} \\
\hspace{-0.3cm} {\small Sumbitted to:} \hspace{5.1cm} {\small Sumbitted by:} \\
\hspace{-0.75cm} Dr. Manish Kumar \hspace{4.5cm} Ajay Kumar\\
}
\institute [MNIT Jaipur] {Department of Mathematics \\ \vspace{0.05cm} Malaviya National Institute of Technology Jaipur} \vspace{-0.1cm}

\usepackage{trial_theme}

% defs
\def\cmd#1{\texttt{\color{red}\footnotesize $\backslash$#1}}
\def\env#1{\texttt{\color{blue}\footnotesize #1}}
\definecolor{deepblue}{rgb}{0,0,0.5}
\definecolor{deepred}{rgb}{0.6,0,0}
\definecolor{deepgreen}{rgb}{0,0.5,0}
\definecolor{halfgray}{gray}{0.55}



\begin{document}
\hspace*{-3cm}    \verb|\rmfamily|: \rmfamily
\begin{frame}[plain]
\vspace{0.2cm}
%\centering { \small \textit{Semester Progess Report} }
\titlepage
%\vspace{0.1cm}
%\begin{center}
% $68^{\text{th}}$ Conference of the ISTAM
%Pre-Ph.D.  Open Seminar  on
% \vspace*{-0.2cm}
%\end{center}
% \titlepage
% \vspace{-4cm}
\end{frame}


 \begin{frame}{Introduction}
     The Navier--Stokes equations describe the motion of viscous, incompressible fluids and
form the cornerstone of fluid dynamics. They model diverse phenomena such as air
flow, ocean currents, and blood circulation. Despite their seemingly simple appearance,
these equations pose deep mathematical challenges in three dimensions. The central
open question is whether smooth, globally defined solutions always exist or whether
singularities points where velocity or pressure become infinite, can develop in finite
time.\\
The problem of proving the existence, uniqueness, and regularity of solutions
in three dimensions is one of the seven Clay Mathematics Institute Millennium Prize
Problems. Solving it would significantly advance our understanding of turbulence and
the mathematical structure of fluid motion.

 \end{frame}
 \begin{frame}{Mathematical Background}
    The incompressible Navier--Stokes equations in vector form are:
\[
\rho\left( \frac{\partial \mathbf{u}}{\partial t} + (\mathbf{u}\cdot\nabla)\mathbf{u} \right)
= -\nabla p + \mu \nabla^{2}\mathbf{u} + \mathbf{f},\\ 
\hspace{5.5cm}\nabla \cdot\mathbf{u}=0,
\]\\
where:
\begin{itemize}
    \item $\mathbf{u}$ -- Velocity vector,
    \item $p$ -- Pressure,
    \item $\rho$ -- Density,
    \item $\mu$ -- Dynamic viscosity,
    \item $\mathbf{f}$ -- External body forces such as gravity.
\end{itemize}  
 \end{frame}
 \begin{frame}{The Main Mathematical Questions Are}{Existence:}
     \textbf  {Do solutions always exist for all time?}\\
We know that solutions exist for a short time (weak solutions).\\ 
But we don't know if a smooth solution continues to exist forever when the fluid becomes very turbulent.\\
    \textbf{The Problem:}\\
    The equations might produce extremely large velocities or infinite values in a finite time.
 \end{frame}
 \begin{frame}{The Main Mathematical Questions Are}{Uniqueness:}
  \textbf{Is the solution one-of-a-kind?}\\
  Even if a solution exists, we do not know whether it is the only solution.\\
In very chaotic flows, it is possible that two different solutions may satisfy the equations with the same starting conditions.\\
\textbf{This is not yet proven.} 
 \end{frame}
 \begin{frame}{The Main Mathematical Questions Are}{Regularity:}
\textbf{ Is the solution always smooth (no infinite spikes or jumps)?}\\
Regularity means:
  \begin{itemize}
      \item No blow-up of velocity.
      \item No infinite energy.
      \item No sudden singularities.
  \end{itemize}
In 3D, it is unknown whether the velocity field always remains smooth.
If smoothness breaks, the solution forms a singularity.    
\end{frame}
\begin{frame}{Why the Problem Is Difficult}
\textbf{The main challenges are:}
\begin{itemize}
    \item The nonlinear convection term $ (\mathbf{u}\cdot\nabla)\mathbf{u}$ can amplify small disturbances.
     \item Vorticity stretching in $3D$ can potentially cause blow-up.
      \item The diffusion term  $ \nu \nabla^{2}\mathbf{u} $ must control the nonlinear growth, but it is not always clear if it succeeds.

\end{itemize}
In $2D$, vorticity cannot be stretched, so solutions stay regular for all time. This difference makes the $3D$ case much harder.
\end{frame}
    \begin{frame}{Known Results}
    \begin{itemize}
        \item \textbf{Local existence:} Smooth solutions exist for short time for any smooth initial data.
         \item \textbf{Global weak solutions (Leray–Hopf):} Proven to exist for all time, but may not be smooth.
         \item \textbf{Energy inequality:} Ensures basic control over weak solutions.
         \end{itemize}
\textbf{Main unknown:} Do these weak solutions become smooth? Or can singularities form?
    
        
    \end{frame}
    \begin{frame}{Singularities}
    \textbf{What Are Singularities?}\\
    
    A singularity occurs when:\\
    \begin{itemize}
        \item Velocity $\lvert u \rvert$ becomes infinite, or
        \item Derivatives like $\nabla u $ blow up, causing infinite energy concentration.
    
      \end{itemize} 
      \textbf{Why Singularities Matter}
      \begin{itemize}
          \item If singularities form, global smooth solutions cannot exist.
          \item Without smoothness, we lose control over the model’s predictability.
          \item The Clay Millennium Prize Problem asks if solutions:
          \begin{itemize}
              \item Stay smooth forever, OR
              \item Become singular in finite time.
          \end{itemize}
      \end{itemize}
    \end{frame}
    \begin{frame}{Why 2D Has No Singularities but 3D Might}
    \textbf{In $2D:$}\\
    \begin{itemize}
        \item vorticity cannot stretch,
        \item stays bounded,
        \item smoothness stays forever,
        \item no singularities.
    \end{itemize}
    \textbf{In $3D:$}\\
    The vorticity stretching mechanism may amplify rotation without bound.  
Mathematically, this effect appears in the nonlinear term
\[
(\boldsymbol{\omega} \cdot \nabla)\mathbf{u},
\]
which represents the stretching of vorticity by the velocity gradient.  
Uncontrolled growth of this term is one possible route to singularity formation.      
    \end{frame}
    \begin{frame}{Smooth Initial Data Hypothesis}

Assume the initial velocity field satisfies
\[
\mathbf{u}_0(x) \in C^{\infty}(\mathbb{R}^3),
\qquad 
\nabla \cdot \mathbf{u}_0 = 0.
\]

\noindent
\textbf{The question:} Does this smoothness persist for all future times?

\medskip

\noindent
\textbf{Reason:}  
\begin{itemize}
    \item If the initial data is already rough, then singularities could appear trivially.
    \item The real challenge is to determine whether initially smooth data can develop a
    finite-time blow-up in the incompressible Navier--Stokes equations.
\end{itemize}
\end{frame}
\begin{frame} {Regularity Criteria Hypothesis}
Assume certain norms of the velocity field remain finite.  
A classical example is the\\
\textbf{Prodi--Serrin condition}:\\

\[
\mathbf{u} \in L^{p}(0,T; L^{q}(\mathbb{R}^3)), 
\qquad 
\frac{2}{p} + \frac{3}{q} = 1,
\qquad 
q > 3,
\]

then the solution remains smooth on the interval \([0,T]\).

\medskip

\noindent
\textbf{Idea:}
     If the velocity field is sufficiently "controlled" in certain space--time norms, then no singularity can form, and the solution stays smooth.
\end{frame}
 \begin{frame}{Energy Inequality Hypothesis (Leray--Hopf Framework)}

Any weak solution must satisfy the energy inequality
\[
\| \mathbf{u}(t) \|_{L^2}^2 
\;+\;
2\nu \int_0^{t} 
\| \nabla \mathbf{u}(s) \|_{L^2}^2 \, ds
\;\le\;
\| \mathbf{u}_0 \|_{L^2}^2 .
\]

\medskip

\noindent
\textbf{Purpose:}
\begin{itemize}
    \item Prevents energy blow-up.
    \item Ensures minimal physical dissipation.
    \item Used to construct global weak (Leray--Hopf) solutions.
\end{itemize}
\end{frame}
\begin{frame}{Conclusion}
    The 3D Navier–Stokes equations pose a major unsolved challenge in mathematics. The core question is whether smooth solutions remain regular for all time or develop singularities. Understanding existence, uniqueness, and regularity would resolve a Millennium Prize problem and deepen our knowledge of turbulence and real fluid behavior.
        
    

 \end{frame}
    \begin{frame}{Active researchers}
    \textbf{Alexey Cheskidov and  Mimi Dai (jun 2025)}\\
    They recently published a paper presenting improved regularity criteria for 3D Navier--Stokes / MHD systems -- giving conditions under which blow-up can be ruled out.    
    \end{frame}
    \begin{frame}{References}
        \begin{itemize}
            \item Leray, J. (1934). “Essai sur le mouvement d’un liquide visqueux emplissant l’espace.” Acta Mathematica, 63(1), 193–248.
            \item Ladyzhenskaya, O. A. (1963). The Mathematical Theory of Viscous  Incompressible Flow. Gordon and Breach.
            \item Serrin, J. (1962). “On the interior regularity of weak solutions of the Navier--Stokesequations.” Archive for Rational Mechanics and Analysis, 9, 187–195.
            \item Caffarelli, L., Kohn, R., and Nirenberg, L. (1982). “Partial regularity of suitable weak solutions of the Navier--Stokes equations.”Communications on Pure and Applied Mathematics, 35(6), 771–831.
        \item Fefferman, C. L. (2006). “Existence and Smoothness of the Navier--Stokes Equation.” The Clay Mathematics Institute Millennium Prize Problem.
        \end{itemize}
    \end{frame}  
    \begin{frame}
    \centering
    {\Huge \textbf{THANK YOU}}
 \end{frame}

\end{document}